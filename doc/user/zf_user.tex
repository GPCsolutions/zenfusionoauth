\documentclass[12pt,a4paper,titlepage]{article}
\usepackage[utf8x]{inputenc}
\usepackage[francais]{babel}
\usepackage[T1]{fontenc}
\usepackage{amsmath}
\usepackage{amsfonts}
\usepackage{amssymb}
\usepackage{makeidx}
\usepackage{graphicx}
\usepackage{hyperref}
\usepackage[left=2cm,right=2cm,top=2cm,bottom=2cm]{geometry}

\hypersetup{colorlinks=true, linkcolor=blue, citecolor=blue, filecolor=blue, urlcolor=blue, pdftitle=Documentation utilisateur ZenFusion, pdfauthor=Sébastien Bodrero Raphaël Doursenaud, pdfsubject=, pdfkeywords=}

\author{Sébastien \textsc{Bodrero}, Raphaël \textsc{Doursenaud}}

\title{ZenFusion™ pour Google Apps™ : Documentation Utilisateur}

\date{\today}

\begin{document}
	
	\maketitle	

\section{Paramétrage}
		\subsection{Pré-requis}
			\begin{itemize}
			\item Un compte \emph{Google Gmail} par utilisateur
			\item Le mot de passe et l'autorisation de chaque utilisateur
			\end{itemize}
		
		\subsection{Paramétrage}
		
			\paragraph{}Ce dernier, relativement simpliste, s'opère dans la fiche personnelle de chaque utilisateur.
			Rendez-vous à l'accueil de \emph{Dolibarr} puis dans le menu \textbf{Utilisateurs \& Groupes->utilisateurs} : 
			
			\paragraph{}
			Effectuez les opérations suivantes :
			
			\begin{itemize}
			\item Sélectionnez l'utilisateur concerné
			\item Rendez-vous dans l'onglet \emph{Google Apps}
			
			\end{itemize}
			
			\paragraph{}
			\textbf{Pour activer le service :}
			
			\begin{itemize}
			\item Cliquez sur le bouton \emph{Activer}
			\end{itemize}
			
			\subparagraph{}
			Une fenêtre sécurisée vous invite à saisir le mot de passe de votre utilisateur autorisant ainsi l'application à accéder aux informations du compte \emph{Gmail} de ce dernier :
			
			\begin{itemize}
			\item Saisissez le mot de passe et cliquer sur \emph{Connexion}
			\item Cliquez sur \emph{Accorder l'accès}
			\end{itemize}
			
			\subparagraph{}
			De retour sur la fiche utilisateur, vous constatez l'activation du service en visionnant la ligne \emph{Accès au service} qui doit être positionnée sur \emph{Activé}.
			
			\paragraph{}
			\textbf{Pour Désactiver le service :}
			
			\begin{itemize}
			\item Cliquez sur le bouton \emph{Désactiver}
			\end{itemize}
			
			\subparagraph{}
			Vous constater la désactivation du service en visionnant la ligne \emph{Accès au service} qui doit être positionnée sur \emph{Désactivé}.
			
\end{document}